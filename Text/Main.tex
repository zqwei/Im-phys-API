% Version Control History
% 
% date: 2015 - 01 - 23 
% created by: Ziqiang Wei
% commented by: 


\documentclass[12pt, oneside]{nature}
%\usepackage{geometry}                		% See geometry.pdf to learn the layout options. There are lots.
%\geometry{letterpaper}                   		% ... or a4paper or a5paper or ... 
%%\geometry{landscape}                		% Activate for for rotated page geometry
%\usepackage[parfill]{parskip}    		        % Activate to begin paragraphs with an empty line rather than an indent					
%\usepackage{amssymb}
\usepackage{amsmath}

\bibliographystyle{naturemag}
\title{Trade off between decodability of stimulus and precision of temporal dynamics 
in different recording methods}
\author{Ziqiang Wei, Tsai-Wen Chen, Nuo Li, Karel Svoboda, Shaul Druckmann}
\date{}

\begin{document}
\maketitle
\begin{affiliations}
 \item Janelia Research Campus, HHMI, Ashburn, VA 20147
 \item Department of Neuroscience, the Johns Hopkins University, Baltimore, MD 21205
\end{affiliations}

\newpage

\begin{abstract}
\end{abstract}

\newpage

% \section*{Results}

\section*{ALM neurons show the similarity choice-specific activity but 
distinguishable temporal dynamics across datasets using different recording techniques}

Mice were trained to discriminate the location of a stimulus onto their
whiskers, held it in the memory, and finally made a licking right or
left response according to their tactile perception of the after a delay.

ALM neurons exhibit the choice-specific activity
in the studies using electrophysiological recordings

(Guo et al., 2014). For example, 


These results are based on the single unit analysis methods: Figure 1-3

\noindent ALM neurons show the similarity choice-specific activity 
across datasets using different recording techniques (Figure 1).
Most of the neurons shows a selectivity of choice for at least 
one epoch (sample, delay, or response) of the task, while a 
fraction of the neurons reveal a change of the selectivity.

\noindent Firing activity in the spiking-neuron dataset displays a 
steep change in time; those in two GCaMP6f datasets vary fast 
but gradually in time; while that in the GCaMP6s dataset shows 
a slow and gradual transition(Figure 1). The same thing holds true for the
selectivity indices (Supplementary Figure 1).

\noindent Firing activities of neurons in the GCaMP6s dataset for 
different trial type maintain highly distinguishable for a longer 
period of time in a task than those in the other three datasets (Figure 1).

\noindent To quantitively illustrate these observations, we measured changes
of the firing activity and selectivities across task epochs. 

\noindent For each dataset, the difference of the firing activity distributions 
for different trial types is limited (Figure 2), however, that could be obvious 
across the different epochs. In general, the distributions become wider and 
flatter in time, which may result from that more neurons exhibit a distinguishable 
firing activity in different trial types (Figure 1A). 

\noindent Across datasets, neurons in GCaMP6s dataset display a wider 
distribution of their activity in all four epochs than the other three datasets. 

\noindent Moreover, these distributions (Figure 2) are nearly identical for the 
two GCaMP6f datasets, which indicates little effect of delay durations of the task 
to recordings.

\noindent Consistently among all datasets, the distribution of ROC indices (Figure 3)
become wider and flatter against time, indicating the number of neurons 
with strong selectivity of choice increases as a function of time.

\noindent The distribution of spiking-neuron dataset is wider and more heavy-tailed 
than that of GCaMP6s dataset from sample to response; both of them are wider than 
those of two GCaMP6f datasets; the distributions of two GCaMP6f datasets are nearly 
identical, despite the lengths of delays.

\noindent The heavy-tailed distribution of spiking-neuron dataset,
comparing to three Ca++ imaging datasets, could result from
its activity distribution (Poisson distribution vs Gaussian distribution).
see also the sparseness of LDA decoder for licking-right
trials (Supplementary Figure S3), which also indicates a
sparse coding for spiking-neuron dataset, while
dense coding for three Ca++ imaging datasets.

\section*{Distinct characteristics among different Ca++ imaging datasets
can be viewed as a trade-off between decoding accuracy of stimulus
and precision of temporal dynamics of the network}

\noindent In three Ca++ imaging datasets, we see that the selectivity
remains high while the dynamics goes slow in GCaMP6s datasets; 
the selectivity changes rapidly and displays more temporal dynamics
in GCaMP6f datasets. We thus want to explore whether there is a trade-off 
between decoding accuracy of stimulus (using a slow, but high
SNR indicator like GCaMP6s)
and precision of temporal dynamics of the network (using a fast, but low
SNR indicator like GCaMP6f): Figure 4-6

\noindent First, we test whether there is any difference of decodabilities of trial
type and epochs in Ca++ imaging datasets using slow and fast temporal dynamics.

\noindent Decodability of trial type fast saturates to nearly one for spiking-neuron and 
GCaMP6s datasets, however, it ramps slowly for two GCaMP6f datasets. (Figure 4A)

\noindent Decodability of trial type decreases dramatically as randomly removing a fraction of neurons 
in the pool of collected neurons in GCaMP6f dataset, but that is less influenced in
spiking-neuron dataset and GCaMP6s dataset (Figure 4B). For spiking-neuron dataset,
this results from that neurons in this dataset employ the sparse coding (Supplementary
Figure 3); \%KO would thus have little influence, unless a key neuron is kicked out.
For Ca++ imaging datasets, neurons of which use a denser LDA decoder, this difference
stems mainly from the ROC distributions of collected neurons (Figure 3B vs Figures 3CD).

\noindent There are main two factors that can account for the distinct ROC distributions. First,
the GCaMP6f dataset has a larger amount of private noise (we will return to this in Figure 6A).
Second, the GCaMP6s dataset has a slower temporal dynamics, which allows the ROC maintains
distinguishable for a longer time (we will test this hypothesis later in Figure 5).

\noindent Decodability of epochs (Figure 4C) has steep changes at the moment of next 
epoch in the spiking-neuron dataset, and GCaMP6f datasets, while that is 
obviously delayed in the GCaMP6s dataset. This could result from the slow 
temporal dynamics of GCaMP6s indicator, which smooths out the transient
change in time.

\noindent The change of trial-type decodability at \%KOs and the delay of 
decoding epoch can therefore be viewed as two signatures to differentiate 
spiking-neuron, GCaMP6s and GCaMP6f datasets.

\noindent Secondly, the change of decodability of stimulus could exclusively results from the
SNRs of the different Ca++ imaging indicators, in the case of which there would
be no trade-off between between decoding accuracy of stimulus
and precision of temporal dynamics of the network. We then further exam whether
change of decodability of stimulus also comes from the slowness of temporal dynamics,
and in which degree such a transient change in time is smoothed out in each Ca++ imaging dataset.
By doing this, we study the dynamics of instantaneous change of LDA direction 
(for licking-right trials) and that of PCA direction (for maximum variance direction). 
We found:

\noindent Instantaneous LDA decoders display a strong similarity within 
epochs, but weak similarity across epochs for all four datasets. Nevertheless,
that in GCaMP6s dataset also shows a strong similarity across epochs (from sample to response),
which could result from the slow dynamics of GCaMP6s indicator (Figure 5A).

\noindent Instantaneous PCA decoders (Figure 5B) display a strong similarity 
exclusively within a single epochs for all four datasets, indicating
we can use PCA direction as a reference to explore the slowness
of temporal dynamics in encoding trial type.

\noindent Figure 5C shows that instantaneous LDA decoder (x-axis direction) 
has a strong similarity with instantaneous PCA decoder (y-axis direction) at 
the current time, while that has a strong similarity with instantaneous PCA 
decoder at the past time for all three Ca++ imaging datasets, which indicating 
a smoothing effect exists in Ca++ imaging datasets.  

\noindent Importantly, this time lag spans about 3.0 sec for GCaMP6s datasets, 
but is less than 0.5 sec for both GCaMP6f datasets.

\noindent All the results in Figures 4A,C -5 hold true for single sessions
in each datasets (Supplementary Figures 2, 4-8).

\noindent We last want to explore how neuronal variability is distributed for
temporal dynamics, stimulus and their interactions, in which way,
we can quantitively study whether there is a trade-off between 
variability of temporal dynamics and that of stimulus across Ca++ imaging
datasets.

\noindent First, a large fraction of the neuronal variability across all datasets
comes from the neuronal private noise (21\% for spiking-neuron dataset; 
15\% for GCaMP6s dataset; 35\% for GCaMP6f dataset).

\noindent Second, The majority of the variability of spiking-neuron dataset comes from the
temporal dynamics, which is captured by the first two components; that of
GCaMP6s dataset is dominated by the stimulus-driven variability, which is
captured by the first component; that of GCaMP6f dataset is dominated by 
the temporal dynamics, which is captured by the first two components.

\noindent Ca++ imaging recordings would thus display a
trade-off between decodability of trial type 
and temporal dynamics.

\section*{Neuronal mechanism accounts for difference among the datasets
using different recording techniques}
\noindent ZW: we may need more discussion about modeling Ca++ imaging dataset
from spiking neurons.

\noindent Figure 7: schematic diagram of model

\noindent Figure 8: model prediction


% \section*{Discussion}


\begin{methods}
\subsection{Neuronal recordings in the tactile discrimination task}
% ZW: I will write a brief version of this session as what I know about
%     the experiment and recording methods
% ZW: NL and TC will work on this subsection in detail
Experimental scheme (A brief version of this)
Emphasize maybe only on the difference from Guo et al., 2013 Neuron;
and that from Guo et al., 2014 Plos One
Spike dataset (Nuo Li); 
Ca++ imaging dataset (Tsai-Wen Chen): 
% ZW: TC added one more section `GCaMP6s' data with long delay
short delay using GCaMP6f, 
short delay using GCaMP6s, 
long delay using GCaMP6f

\subsection{Preprocessing}
The raw input to analyses of Ca++ imaging datasets
was the time-serial fluorescence of neuronal activity, $F(t)$. 
These fluorescence signals
were converted into $\Delta F/F$, namely the neuronal 
firing activity measured by Ca++ imagings, as

\begin{equation*}
  r(t) = (F(t)-<F(t)>_t)/(<F(t)_t>+c)
\end{equation*}
where $<F(t)>_t$ is mean firing activity of each registered
neuron, and a small constant $c$ prevents division by zero.

In the single-unit analysis of all datasets, we removed the recording units
with the number of correct trials less than 30 in either trial type condition (i.e.
licking right or left). We then applied the single neuron data analyses, like
PSTH, firing-activity distribution, ROC-index distribution and two-sample
z-score analysis of selectivity, to such datasets. The simultaneous recording
datasets were then determined according to the analyses of firing-activity
distribution and ROC-index.



\subsection{Neuronal model -- generation of Ca++ imaging data from spikes}
% ZW: following Spike_Ca_model codes by T-WC on 2014 - 09 - 26
In light of the simultaneously recording of the 

\subsection{Choice index analysis of single neuron} 

\subsection{Population analysis}
Two types 

\subsection{Sparseness of the optimal linear decoder}
We measured the sparseness of the 

\subsection{De-mixed PCA analysis}

\end{methods}

%\newpage
%
%\begin{thebibliography}{1}
%\bibitem{} 
%\end{thebibliography}
%
%\newpage
%
%\begin{addendum}
%	\item [Acknowledgements] 
%	\item [Author contributions] K.S. and S.D. conceived of the project. N.L., T.-W.C. and K.S. developed the animal experiment and recoding methods. N.L. collected the spiking data. T.-W.C. collected the GCaMP6s and GCaMP6f data. T.-W.C. and K.S. developed the experimental scheme of simultaneous recording of spikes and Ca++ imaging signal in \textbf{Figure 7}. T.-W.C. collected the data in \textbf{Figure 7}. Z.W. and S.D. developed analysis methods of neuronal data. Z.W. and T.-W.C. developed neuronal model of Ca++ imaging generation from spiking data. Z.W. performed the analysis and modeling, and contributed to both codes.
%	\item[Competing Interests] The authors declare that they have no
%	competing financial interests.
%	\item[Correspondence] Correspondence and requests for materials
%	should be addressed to ~(email: ).
%\end{addendum}

\newpage

\section*{Figure legends}
\subsection{Figure 1: ALM neurons show the similarity choice-specific activity 
across datasets using different recording techniques.} 
(\textbf{A}) Left panel: normalized neuronal activity of all the neurons in
the spiking-neuron dataset ($n=595$) during a stimulus location discrimination 
task (Guo et al., 2014); Right panel: Exemplary neurons; blue lines
and shadows are for the average and sem of firing activity in licking-right trials; 
red lines and shadows are for the average and sem of firing activity 
in licking-left trials. (\textbf{B-D}) The same plots as \textbf{A}, expect the data 
were collected using Ca++ imaging methods where the indicators were GCaMP6s 
($n=307$), GCaMP6f ($n=114$), and GCaMP6f 
($n=263$), respectively; the delay durations were 1.4 sec, 1.4 sec and 3.0 sec, 
respectively. Time zeros of all trials are aligned to the moment of response cue. 
In all datasets, the neurons exhibit similar heterogeneous activity patterns in a task,
e.g. most of the neurons shows a selectivity of choice for at least one epoch
(sample, delay, or response) of the task, while a fraction of the neurons reveal
a change of the selectivity (which is transient in the spike-neuron dataset, and is
fast but gradual in two GCaMP6s dataset, and is quite slow in the GCaMP6s dataset;
see also \textbf{Supplementary Figure 1} for a detailed demonstration of selectivity
indices against time for neurons in each dataset).
Generally, firing activity in the spiking-neuron dataset displays a 
steep change in time; those in two GCaMP6f datasets vary fast but gradually in
time; while that in the GCaMP6s dataset shows a slow and gradual transition.
On the other hand, firing activities of neurons in the GCaMP6s dataset for 
different trial type maintain highly distinguishable for a longer period of time in a task 
than those in the other three datasets.

\subsection{Figure 2: Distribution of neuronal activities at different epochs of a task.}
Distributions of neuronal activities at four epochs of a task (pre-sample,
sample, delay and response) for four datasets (\textbf{A}: spiking-neuron
dataset; \textbf{B}: GCaMP6s dataset with delay duration of 1.4 sec; 
\textbf{C}: GCaMP6f dataset with delay duration of 1.4 sec; 
\textbf{D}: GCaMP6s dataset with delay duration of 3.0 sec); 
blue bars and fitting lines for licking-right
trials; red bars and fitting lines for licking-left trials; Poisson distribution is fitted
for neurons in the spiking dataset; Gaussian distributions are fitted for those in the other three datasets. 
For each dataset, the difference of
the distributions for different trial types is limited, however, that could be obvious across
the different epochs. In general, the distributions become wider and flatter in
time, which may result from that more neurons exhibit a distinguishable firing
activity in different trial types. Specifically, neurons in GCaMP6s dataset display
a wider distribution of their activity in all four epochs than the other three datasets.
Moreover, these distributions are nearly identical for the two
GCaMP6f datasets, which indicates little effect of delay durations of the task to recordings.

\subsection{Figure 3: Single neuron choice probability index distribution indicates stronger
selectivity of trial type in spiking-neuron and GCaMP6s datasets.}
Distributions of ROC indices for licking-right trials at different epochs of a task
(pre-sample, sample, delay and response) for four datasets (\textbf{A}: spiking-neuron
dataset; \textbf{B}: GCaMP6s dataset with delay duration of 1.4 sec; 
\textbf{C}: GCaMP6f dataset with delay duration of 1.4 sec; 
\textbf{D}: GCaMP6s dataset with delay duration of 3.0 sec); Gaussian distributions are applied
to fit for all four datasets. Consistently among all datasets, the distribution of ROC indices
become wider and flatter against time, indicating the number of neurons with strong
selectivity of choice increases as a function of time. The distribution of spiking-neuron dataset
is wider and more heavy-tailed than that of GCaMP6s dataset from sample to response;
%(another view of this heavy-tailed distribution can be indicated from the spareness of LDA
% coefficients, Supplementary Figure 3); 
both of them are wider than those of two GCaMP6f datasets; the distributions of two GCaMP6f datasets
are nearly identical, despite the lengths of delays.

\subsection{Figure 4: Decodability of trial types and that of task epochs can be viewed as signatures
of datasets.}
(\textbf{A}) Decodability of trial types (LDA decoder) increases as a function of time from sample to response for 
collected neurons in four datasets (from left to right: spiking-neuron dataset, 
GCaMP6s dataset with delay duration of 1.4 sec,
GCaMP6f dataset with delay duration of 1.4 sec, and GCaMP6f dataset with delay duration of 3.0 sec;
the same plots for each simultaneously recording session are shown in \textbf{Supplementary Figure 2}).
Decodability fast saturates to nearly one for spiking-neuron and GCaMP6s datasets, however, it ramps
slowly for two GCaMP6f datasets. 
(\textbf{B}) Decodability decreases as randomly removing a fraction of neurons (a.k.a. \%KO) in 
the pool of collected neurons for each dataset
(from left to right: spiking-neuron dataset, 
GCaMP6s dataset with delay duration of 1.4 sec,
and GCaMP6f dataset with delay duration of 3.0 sec),
which also changes as a function of time. \%KO has a considerable influence on GCaMP6f dataset, while
a very limited effect on the other two. 
(\textbf{C}) Decodability of epochs in a trial varies a function of time of collect neurons
for each dataset
(from left to right: spiking-neuron dataset, 
GCaMP6s dataset with delay duration of 1.4 sec,
GCaMP6f dataset with delay duration of 1.4 sec, 
and GCaMP6f dataset with delay duration of 3.0 sec;
the same plots for each simultaneously recording session are shown in \textbf{Supplementary Figure 4}). 
Decodability of epochs has steep changes at the moment of next epoch
in the spiking-neuron dataset, and GCaMP6f datasets, while that 
is obviously delayed in the GCaMP6s dataset.
The change of trial-type decodability at \%KOs and 
the delay of decoding epoch can therefore be viewed as two signatures
to differentiate spiking-neuron, GCaMP6s and GCaMP6f datasets.

\subsection{Figure 5: Instantaneous change of LDA direction is lagged to that of PCA direction
in Ca++ image datasets.}
(\textbf{A}) Instantaneous LDA decoders (for licking-right trials) display a strong similarity within 
epochs, but weak similarity across epochs for all four datasets (from top to bottom:
spiking-neuron dataset, 
GCaMP6s dataset with delay duration of 1.4 sec,
GCaMP6f dataset with delay duration of 1.4 sec, and GCaMP6f dataset with delay duration of 3.0 sec;
the same plots for each simultaneously recording session are shown in \textbf{Supplementary Figure 6}).
Notably, that in GCaMP6s dataset also shows a strong similarity across epochs (from sample to response).
(\textbf{B}) Instantaneous PCA decoders (for maximum variance direction) display a strong similarity 
exclusively within a single epochs for all four datasets (from top to bottom:
spiking-neuron dataset, 
GCaMP6s dataset with delay duration of 1.4 sec,
GCaMP6f dataset with delay duration of 1.4 sec, and GCaMP6f dataset with delay duration of 3.0 sec;
the same plots for each simultaneously recording session are shown in \textbf{Supplementary Figure 7}).
(\textbf{C}) Similarity of instantaneous LDA decoder (x-axis direction) and instantaneous PCA decoder
(y-axis direction) changes as a function of time(from top to bottom:
spiking-neuron dataset, 
GCaMP6s dataset with delay duration of 1.4 sec,
GCaMP6f dataset with delay duration of 1.4 sec, and GCaMP6f dataset with delay duration of 3.0 sec;
the same plots for each simultaneously recording session are shown in \textbf{Supplementary Figure 8}).
 Spiking-neuron dataset only reveals
strong similarity exclusively within a single epoch, while all three Ca++ imaging dataset
show delayed similarities, i.e. instantaneous change of LDA direction is lagged to that of PCA direction.
Notably, this time lag spans about 3.0 sec for GCaMP6s datasets, but is less than 0.5 sec for both
GCaMP6f datasets. 
Importantly, All results in \textbf{Figures A-C} hold true for single sessions (\textbf{Supplementary Figure 6-8}).

\subsection{Figure 6: Ca++ imaging recordings display a
trade-off between decodability of trial type and temporal dynamics.} 
dPCA analysis on collected neurons in three datasets (from left to right:
spiking-neuron dataset, 
GCaMP6s dataset with delay duration of 1.4 sec,
and GCaMP6f dataset with delay duration of 3.0 sec
): 15 components are applied to account for variability of
stimulus (Stim), temporal dynamics (Time), and their interaction term (Inter).
(\textbf{A}) Percentage of different variabilities captured by different component
(bar plots); the pie plots show the percentage of different variabilities for the
real data. The majority of the variability of spiking-neuron dataset comes from the
temporal dynamics, which is captured by the first two components; that of
GCaMP6s dataset is dominated by the stimulus-driven variability, which is
captured by the first component; that of GCaMP6f dataset is dominated by 
the temporal dynamics, which is captured by the first two components.
Ca++ imaging recordings would thus display a
trade-off between decodability of trial type 
and temporal dynamics.
 (\textbf{B}) The dynamics of the components capturing the maximum
variabilities of interaction term (top), stimulus (middle), and temporal dynamics (bottom).

\subsection{Supplementary Figure 1: Selectivity indices (two-sample
z-score) of licking right trials for four datasets.}
(\textbf{A}) spiking-neuron dataset; 
(\textbf{B}) GCaMP6s dataset with delay duration of 1.4 sec;
(\textbf{C}) GCaMP6f dataset with delay duration of 1.4 sec;
(\textbf{D}) GCaMP6f dataset with delay duration of 3.0 sec.
 

\subsection{Supplementary Figure 2: Decodability of trial types (LDA decoder) increases 
as a function of time from sample to response for 
simultaneously recording sessions in four datasets.} 
(\textbf{A}) spiking-neuron dataset; 
(\textbf{B}) GCaMP6s dataset with delay duration of 1.4 sec;
(\textbf{C}) GCaMP6f dataset with delay duration of 1.4 sec;
(\textbf{D}) GCaMP6f dataset with delay duration of 3.0 sec.

\subsection{Supplementary Figure 3: Kurtosis of LDA decoder
for licking-right trials changes as a function of time
for collected neurons in four datasets.} 
(\textbf{A}) spiking-neuron dataset; 
(\textbf{B}) GCaMP6s dataset with delay duration of 1.4 sec;
(\textbf{C}) GCaMP6f dataset with delay duration of 1.4 sec;
(\textbf{D}) GCaMP6f dataset with delay duration of 3.0 sec.
Comparing the three Ca++ imaging datasets,
spiking-neuron dataset shows an extremely high kurtosis
of LDA decoders in time, indicating that neurons in
spiking-neuron dataset (Ca++ imaging datasets, respectively)
tends to use sparse coding (dense coding, respectively)
of the stimulus. That could result from that the
activity distribution of spiking-neuron dataset
is more heavy-tailed than that of Ca++ imaging 
dataset (Poisson distribution vs Gaussian distribution). One can thus expect to see decodability
is less influenced by \%KO neurons in spiking-neuron dataset
than that in Ca++ imaging datasets.

\subsection{Supplementary Figure 4: Decodability of epochs in a trial varies a function 
of time for simultaneously recording sessions in each dataset using single LDA decoder
for four epochs.} 
(\textbf{A}) spiking-neuron dataset; 
(\textbf{B}) GCaMP6s dataset with delay duration of 1.4 sec;
(\textbf{C}) GCaMP6f dataset with delay duration of 1.4 sec;
(\textbf{D}) GCaMP6f dataset with delay duration of 3.0 sec.

\subsection{Supplementary Figure 5: .Decodability of epochs in a trial varies a function 
of time for simultaneously recording sessions in each dataset using three LDA decoders,
each of which decodes the adjacent epochs.} 
(\textbf{A}) spiking-neuron dataset; 
(\textbf{B}) GCaMP6s dataset with delay duration of 1.4 sec;
(\textbf{C}) GCaMP6f dataset with delay duration of 1.4 sec;
(\textbf{D}) GCaMP6f dataset with delay duration of 3.0 sec.
Comparing to the three Ca++ imaging datasets, neurons in spiking-neuron
dataset display a discontinuity and transient change of selectivity in time; while
both GCaMP6f datasets reveal a faster change of selectivity than 
the GCaMP6s dataset in time.

\subsection{Supplementary Figure 6: Instantaneous LDA decoders (for licking-right trials) 
display a strong similarity within epochs for all simultaneously recording sessions in four datasets.} 
(\textbf{A}) spiking-neuron dataset; 
(\textbf{B}) GCaMP6s dataset with delay duration of 1.4 sec;
(\textbf{C}) GCaMP6f dataset with delay duration of 1.4 sec;
(\textbf{D}) GCaMP6f dataset with delay duration of 3.0 sec.

\subsection{Supplementary Figure 7: Instantaneous PCA decoders 
(for maximum variance direction) display a strong similarity 
exclusively within single epochs for all simultaneously recording sessions
in four datasets.} 
(\textbf{A}) spiking-neuron dataset; 
(\textbf{B}) GCaMP6s dataset with delay duration of 1.4 sec;
(\textbf{C}) GCaMP6f dataset with delay duration of 1.4 sec;
(\textbf{D}) GCaMP6f dataset with delay duration of 3.0 sec.

\subsection{Supplementary Figure 8: Similarity of instantaneous LDA decoder 
(x-axis direction) and instantaneous PCA decoder
(y-axis direction) changes as a function of time 
for all simultaneously recording sessions
in four datasets.} 
(from top to bottom:
(\textbf{A}) spiking-neuron dataset; 
(\textbf{B}) GCaMP6s dataset with delay duration of 1.4 sec;
(\textbf{C}) GCaMP6f dataset with delay duration of 1.4 sec;
(\textbf{D}) GCaMP6f dataset with delay duration of 3.0 sec.


\end{document}  
